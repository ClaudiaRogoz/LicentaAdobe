%!TEX program = xelatex
\documentclass[9pt, compress]{beamer}
\usetheme[titleprogressbar]{m}

\usepackage{booktabs}  
\usepackage[scale=2]{ccicons}
\usepackage{minted}
\usepgfplotslibrary{dateplot}
\usemintedstyle{trac}
\author{\textbf{Rania Sayed}, \textbf{Hazem Al Saied} } 
\title{The research system in Germany}
%\subtitle{}
%\logo{}
\institute{\textbf{Uinversité de Lorraine}}
\date{October 2015}
%\subject{}
%\setbeamercovered{transparent}
%\setbeamertemplate{navigation symbols}{}
\begin{document}
    \maketitle
    \begin{frame}
        \frametitle{Outlines}
        \tableofcontents{}
    \end{frame}
\section{Introduction}
    \begin{frame}
        \frametitle{Introduction}
        \begin{itemize}
            \item Germany has one of the largest research systems in the OECD
            \item there are nearly 1,000 public   funded institutions of science, research and development in Germany
             
        \end{itemize}
    \end{frame}

    \begin{frame} %BilantInfo
        \frametitle{Introduction 2}
        \includegraphics[width=\textwidth,height=300pt]{img/no_researchers.jpg}
        
    \end{frame}
    \section{Overview of the German research system}
    \begin{frame} 
        \frametitle{Overview of the German research system}
        \begin{figure}
        \includegraphics[width=\textwidth,height=150pt]{img/R_org_graph.jpg}
        \caption{Distribution of budget on different sectors}
        \end{figure}
    \end{frame}
    
    \begin{frame} 
        \frametitle{Overview of the German research system}
        
            Public funding for different aspects of research is organised in one of three ways: 
            \begin{itemize}
            \item From federal sources (e.g. project funding); 
            \item From state sources (e.g. institutional funding for higher education
institutions, state R\&D institutions); 
            \item Jointly from federal and state sources according to agreed formula (e.g.institutional funding of public research institutes of national significance; project funding for universities;categories of research infrastructure).
        \end{itemize}
    
    \end{frame}
    \section{Research Organisations}
    \begin{frame} 
        \frametitle{Research Organisations}
        \begin{figure}
        \includegraphics[width=\textwidth,height=175pt]{img/ResearchOrg.jpg}
        \caption{Overview of research-performing organisations in Germany}
        \end{figure}
    \end{frame}
    \subsection{Public research institutes}
    \begin{frame} 
        \frametitle{Public research institutes}
        
            Public research institutes are  organised into four large networks:
            \begin{itemize}
            \item The Max Planck Gesellschaft (MPG) 
            \item The Fraunhofer Gesellschaft (FhG)
            \item The Helmholtz-Gemeinschaft Deutscher Forschungszentren (HGF) 
            \item The Wissenschaftsgemeinschaft Wilhelm-Gottfried-Leibniz (WGL)
        \end{itemize}
    \end{frame}
    \subsection{Research Activities}
    \begin{frame} 
            \frametitle{Research Activities}
        
            \centering
        \includegraphics[scale=0.7]{img/Domain_Res.jpg}
    \end{frame}

    \section{Research Funding System }
    \begin{frame} 
        \frametitle{Funding types in Germany}
        \begin{itemize}
            \item\textbf{ Government Funding}
                \begin{itemize}
                    \item more than 390 public universities and colleges are funded by the Länder.
                \end{itemize}
        \end{itemize}
    
            \begin{itemize}
                \item\textbf{ The European Funding}
            \end{itemize}
            \begin{itemize}
                \item\textbf{ The Industrial Funding}
                            \begin{itemize}
                                \item  In 2012, German companies  spending amounted to a total of 2.5 billion euros.
                            \end{itemize}
            \end{itemize}

    \end{frame}
    \begin{frame} 
        \includegraphics[width=\textwidth,height=175pt]{img/FundingoResearch.png}
        \newline
        \centering
        Figure Shows the collaboration between the federal and states governments.
    \end{frame}

    \section{PHD Student }
    \begin{frame} 
        \frametitle{PHD Student}
        \begin{quotation}
            4,000 international graduates complete their doctorate in Germany every year.
        \end{quotation}
        \begin{itemize}
            \item five reasons to do your PhD in Germany:
            \begin{enumerate}
                \setcounter{enumi}{0}
                \item Outstanding reputation of German doctorate
                \item Strong international focus: PhD in English
                \item Good funding opportunities 
                \item Excellent research infrastructure
                \item High standard of living           
            \end{enumerate}
        \end{itemize}
    \end{frame}
    \subsection{Ways to do th PHD in Germany}
    \begin{frame} 
        \frametitle{Ways to do the PHD in Germany}
        \begin{enumerate}
            \setcounter{enumi}{0}
            \item \textbf{Individual Doctorate}
            \begin{itemize}
                \item Involves the thesis produced under the supervision of a professor. 
                \item Three to five years are normal. 
            \end{itemize}
            \item \textbf{Structured PhD Programs}
            \begin{itemize}
                \item A team of supervisors look after a group of doctoral students. 
                \item The duration of  studies is  limited to three years.
            \end{itemize}
        \end{enumerate} 
    \end{frame}
    \subsection{Computer science} 
    \begin{frame} 
        \frametitle{Computer science}
        programs and organizations  each PhD student should know :
        \begin{enumerate}
            \setcounter{enumi}{0}
            \item \textbf{Bitkom}
            \begin{itemize}
                \item  Represent more than 2,300 companies in the digital economy.
            \end{itemize}
            \item \textbf{BMBF} 
            \begin{itemize}
                \item Drives more than 80 per cent of innovations in Germany.
            \end{itemize}
            \item \textbf{Action Program iD2010}
            \begin{itemize}
                \item Coordinates funding programs, to ensure that the information society can develop further.
            \end{itemize}
        \end{enumerate}
    \end{frame}
    \subsection{Funding PhD in Germany}
    \begin{frame} 
        \frametitle{Funding PhD in Germany: Costs of Study}
        \includegraphics[width=\textwidth,height=150pt]{img/cost.png}
        \begin{center}
            How much “normal students” spend per month in Germany.
        \end{center}
    \end{frame}
    \begin{frame} 
        \frametitle{Funding Models}
    \begin{quotation}
        In 2012, DAAD supported over 4,700 international doctoral students in Germany with scholarships.
    \end{quotation}
        \begin{enumerate}
            \setcounter{enumi}{0}
            \item \textbf{SCHOLARSHIPS}
            \item \textbf{RESEARCH ASSOCIATE JOBS}
            \begin{itemize}
                \item As a rule, doctoral candidates work at the chair of their supervising professor as research associates with temporary part-time contracts.
            \end{itemize}
            \item  \textbf{SIDE JOBS OUTSIDE RESEARCH}
        \end{enumerate} 
    \end{frame}
    \begin{frame} 
        \frametitle{Funding Databases}
        \begin{enumerate}
            \setcounter{enumi}{0}
            \item \textbf{DAAD Scholarship Database}
            \begin{itemize}
                \item German Academic Exchange Service (DAAD) is the largest awarder of scholarships. 
            \end{itemize}
            \item \textbf{EURAXESS Funding Database}
            \begin{itemize}
                \item  Comprises more than 100 programs offered by funding organizations in Germany.
            \end{itemize}
            \item  \textbf{Stipendienlotse}
            \begin{itemize}
                \item Is the database of the Federal Ministry of Education and Research (BMBF) 
            \end{itemize}
        \end{enumerate} 
    \end{frame}
    \section{Conclusion}
    \begin{frame} 
        \frametitle{Conclusion}
        \begin{itemize}
            \item Research in Germany is characterised by an excellent infrastructure, a wide variety of disciplines, well-equipped research facilities and competent staff.\\
            \item Germany offers various forms of research locations: universities, non-university institutes, companies and institutions run by federal or state (“Länder”) authorities.\\
            \item There are more than 800 publicly funded research institutions in Germany, plus research and development (R\&D) centres run by companies.
        \end{itemize}
    \end{frame}
\end{document}
